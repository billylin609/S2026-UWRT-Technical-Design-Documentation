\section{Introduction}
\label{sec:intro}

This documentation provides an in-depth technical review of the UW Robotics team's development process for our new rover platform, Sparky. The objectives of this documentation are as follows:

\begin{enumerate}
    \item Provide in-depth analysis of each competition mission
    \item Evaluate possible technologies and solutions proposed to address each mission
    \item Document the rover design, system characterization, and design variants
    \item Provide a reference manual for future UW Robotics members to reproduce the platform with a comprehensive understanding of system capabilities
\end{enumerate}

This document is organized into the following sections:

\begin{itemize}
    \item Problem Definition
    \begin{itemize}
        \item Mission Requirements
        \item Mission Analysis
        \item Solution Proposal
    \end{itemize}
    \item Environment Setup
    \begin{itemize}
        \item Development and DevOps Setup
        \item System Setup
        \item Testing and Validation
    \end{itemize}
    \item System Design
    \begin{itemize}
        \item Baseline Performance
        \item Design Notes
        \item Implementation Details
    \end{itemize}
    \item Maintenance Log
\end{itemize}

\textbf{All systems are deployed with appropriate isolation.} We emphasize utilizing technologies such as \texttt{python-venv} and \texttt{conda}. Additionally, following the practices established in existing Python projects, project folders should adhere to the template structure shown in Figure \ref{fig:project_structure}.

\begin{figure}[htbp]
    \centering
    \begin{mdframed}
        \dirtree{%
            .1 project/.
            .2 .git/workflows/.
            .2 src/.
            .3 nodes/ \quad (ROS nodes).
            .3 lib/ \quad (algorithms).
            .2 inc/ \quad (C++ headers).
            .2 mod/ \quad (Python modules).
            .2 common/ \quad (baseline implementations \& metrics).
            .2 config/ \quad (YAML/JSON configuration).
            .2 tests/ \quad (unit \& integration tests).
            .2 thirdparty/ \quad (ported dependencies).
            .2 docs/.
            .2 launch/ \quad (ROS launch files).
            .2 .gitignore.
            .2 Makefile, CMakeLists.txt, package.xml, setup.py.
            .2 README.md.
        }
    \end{mdframed}
    \caption{Project structure for UWaterloo robotics projects. This structure derived from standard ROS conventions with support for mixed Python and C++ development for any projects.}
    \label{fig:project_structure}
\end{figure}

In summary, this documentation serves as a knowledge base and reference for future team members interested in contributing to the project.